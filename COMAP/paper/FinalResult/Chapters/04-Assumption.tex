% reasonable Assumptions
\section{Assumptions}
% reasonable Assumptions
\section{Assumptions and Justifications}
After analyzing the problem, we develop basic assumptions about the solution to the problem and the model to be used, as follows.

\subsection{Assumption 1}

\textbf{Developing a method to measure “momentum” could help capture the flow of the game and quantify player performance.}
\par
Justification: The importance of momentum is mentioned countless times in competitive sports. In team sports, poise affects the chemistry between teammates; in individual sports, poise can have an impact on self-confidence. Therefore, a reasonable quantification of momentum can effectively help us to judge the game situation.

\subsection{Assumption 2}
\textbf{Developing an effective model of swing prediction needs to make use of the information earlier.}
\par
Justification: Tennis is a test of physical and mental strength, and the course of the match depends not only on the current state of play but also on all aspects of the players' competition in the previous sets and games. Therefore, we believe that to develop a successful model, we need to take into account the previous match conditions.

\subsection{Assumption 3}
\textbf{Not taking weather into account.}
\par
Justification: The weather factor affects players on the same field in a similar way, and play is suspended in the event of inclement weather.