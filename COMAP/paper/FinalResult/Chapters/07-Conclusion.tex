\section{Conclusion}
\subsection{Result of Problem 1}
We measured the factors that affect athletes' momentum through the AHP and TOPSIS methods, and found that momentum can, to a large extent, measure athletes' performance and predict the results of the game; we created a visualization chart of athlete momentum for the game mentioned in the title, which can be seen in Figure \ref{fig:M3}
\subsection{Result of Problem 2}
By comparing the momentum of both players to determine the winners and losers, the accuracy reaches more than 70\% in some cases. We also achieved an average prediction accuracy of 65\% in the LSTM model we developed. \par
Based on the above data, we believe that momentum plays a role in the game. 
\subsection{Result of Problem 3}
In order to fully utilize the data and improve the robustness of the model, we divided all the matches provided by the problem into a training set and a test set, and evaluated the model performance by using the average accuracy of the test set as a criterion. In the decision tree model we developed, we achieved a turning point prediction success rate of about 75\% and obtained major indicators. Next in our improved LSTM model using indicators, the best average prediction accuracy reached 81\%. \par
Regarding coaching, we conclude that essential indicators can affect the game and cause swing. When a player participates in a new tournament, we recommend focusing on these main factors and design tactics with his characteristics. Maximize the duration of his side's advantage and minimize fluctuations in the game that are not in his favor. \par
\subsection{Result of Problem 4}
\subsubsection{Model Expansion}
Regarding the fluctuations of the model, since there are many factors that affect tennis matches, we believe that the data given in the question is not comprehensive enough in the prediction process. We believe that it may be necessary to add factors such as the athlete's age, injuries, on-field tactics and psychological quality for a more comprehensive analysis to predict.
\subsubsection{Model Generalization}
When our model is extended to tennis matches, it generally predicts better, but there are still fluctuations. For games with poor predictions, we found that it was due to factors such as game rules, different court lengths, and other factors that did not match our original data. When the model is extended to other competitions, the prediction performance is poor because other competitions have different emphasis on measuring athlete performance.
